\documentclass[12pt]{extarticle}
%Some packages I commonly use.
\usepackage[portuguese]{babel}
\usepackage{graphicx}
\usepackage{framed}
\usepackage[normalem]{ulem}
\usepackage{amsmath}
\usepackage{amsthm}
\usepackage{amssymb}
\usepackage{amsfonts}
\usepackage{enumerate}
\usepackage[utf8]{inputenc}
\usepackage{float}
\usepackage{gensymb}
\usepackage[top=1 in,bottom=1in, left=1 in, right=1 in]{geometry}
\usepackage{multirow}
\usepackage{caption}
\usepackage{subcaption}
\usepackage[utf8]{inputenc}
\usepackage{tikz}

%A bunch of definitions that make my life easier
\newcommand{\matlab}{{\sc Matlab} }
\newcommand{\cvec}[1]{{\mathbf #1}}
\newcommand{\rvec}[1]{\vec{\mathbf #1}}
\newcommand{\ihat}{\hat{\textbf{\i}}}
\newcommand{\jhat}{\hat{\textbf{\j}}}
\newcommand{\khat}{\hat{\textbf{k}}}
\newcommand{\minor}{{\rm minor}}
\newcommand{\trace}{{\rm trace}}
\newcommand{\spn}{{\rm Span}}
\newcommand{\rem}{{\rm rem}}
\newcommand{\ran}{{\rm range}}
\newcommand{\range}{{\rm range}}
\newcommand{\mdiv}{{\rm div}}
\newcommand{\proj}{{\rm proj}}
\newcommand{\R}{\mathbb{R}}
\newcommand{\N}{\mathbb{N}}
\newcommand{\Q}{\mathbb{Q}}
\newcommand{\Z}{\mathbb{Z}}
\newcommand{\<}{\langle}
\renewcommand{\>}{\rangle}
\renewcommand{\emptyset}{\varnothing}
\newcommand{\attn}[1]{\textbf{#1}}
\theoremstyle{definition}
\newtheorem{theorem}{Theorem}
\newtheorem{corollary}{Corollary}
\newtheorem*{definition}{Definition}
\newtheorem*{example}{Example}
\newtheorem*{note}{Note}
\newtheorem{exercise}{Exercise}
\newcommand{\bproof}{\bigskip {\bf Proof. }}
\newcommand{\eproof}{\hfill\qedsymbol}
\newcommand{\Disp}{\displaystyle}
\newcommand{\qe}{\hfill\(\bigtriangledown\)}
\setlength{\columnseprule}{1 pt}
\usepackage[utf8]{inputenc}

\usetikzlibrary{arrows,shapes,positioning,angles}
\usetikzlibrary{decorations.markings}
\tikzstyle arrowstyle=[scale=1]
\tikzstyle directed=[postaction={decorate,decoration={markings,
    mark=at position 1 with {\arrow[arrowstyle]{stealth}}}}]
\tikzstyle reverse directed=[postaction={decorate,decoration={markings,
    mark=at position 1 with {\arrowreversed[arrowstyle]{stealth};}}}]

\title{Aula 21 - Força de Atrito ($\vec{F}_{at}$)}
\author{Felipe Salvador}
\date{Atualizado em \today}

\begin{document}

\maketitle

\section{Introdução}
A força de atrito é uma força de contato quando um corpo rugoso tenta deslizar sobre outro objeto rugoso. Essa força é uma força contrária à tendência de movimento ou ao movimento e ela é gerada pela interação eletrostática dos átomos nas camadas mais externas entre os objetos em contato.

Há 2 tipos de atrito:
\begin{itemize}
    \item \textbf{Atrito Estático} - é a força de atrito de quando o um corpo está parado em relação ao outro e o atrito é o que mantém os dois corpos parados;
    \item \textbf{Atrito Dinâmico} - é o atrito de quando um corpo desliza sobre outro corpo.
\end{itemize}

\section{Atrito Estático}
O atrito é estático quando o objeto não desliza sobre outro objeto ou chão:
\begin{figure}[H]
    \centering
    \begin{tikzpicture}
         \coordinate (O) at (0,0);
         \coordinate (A) at (-3,0);
         \coordinate (B) at (5,0);
         \coordinate (C) at (3,2);
         
         \draw[black,ultra thick] (A) -- (B);
         \draw[black,thick] (O) rectangle (C);
         \draw[black,ultra thick,->] (3,1) -- ++(2,0);
         \draw[black,ultra thick,->] (O)+(0,.5) -- ++(-1.5,.5);
         
         \path (O) ++ (1.5,2.2) node {$v=0\,m/s$};
         \path (O) ++ (-0.7,0.9) node {$\vec{F}_{at}$};
         \path (C) ++ (1,-0.6) node {$\vec{F}$};
         \path (O) ++(1.5,1) node {$m$};
    \end{tikzpicture}
    \caption{Uma força aplicada no bloco e o atrito reagindo a força aplicada.}
    \label{fig:atrito_estat}
\end{figure}

Como o bloco está parado e se mantém parado, então a soma de forças aplicada nele tem que ser zero. Logo:
\begin{equation}
    F - F_{at} = 0 \implies \boxed{F_{at}=F}
\end{equation}
\textbf{Ou seja, no caso do atrito estático (bloco parada), a força de atrito, nesse caso é igual à força aplicada.}

Há um caso, que é o \textbf{atrito estático limite}, que é o atrito máximo gerado no caso estático, ou seja, caso uma força seja levemente maior que esse atrito estático limite, o corpo entra em movimento.

\begin{figure}[H]
    \centering
    \begin{tikzpicture}
         \coordinate (O) at (0,0);
         \coordinate (A) at (-3,0);
         \coordinate (B) at (5,0);
         \coordinate (C) at (3,2);
         
         \draw[black,ultra thick] (A) -- (B);
         \draw[black,thick] (O) rectangle (C);
         \draw[black,ultra thick,->] (3,1) -- ++(2,0);
         \draw[black,ultra thick,->] (O)+(0,.5) -- ++(-1.5,.5);
         \draw[black, ultra thick, ->] (O)+(1.5,0) --
         ++(1.5,-1.5);
         \draw[black, ultra thick, ->] (O)+(1.5,2) -- ++(1.5,3.5);
         \draw[black,thick] (3.4,3) rectangle (5.5,3.6);
         
         \path (O) ++ (4.5,3.2) node {$v=0\,m/s$};
         \path (O) ++ (1.5,3.9) node {$\vec{N}$};
         \path (O) ++ (1.5,-1.9) node {$\vec{P}$};
         \path (O) ++ (-0.9,1.1) node {$\left(\vec{F}_{at}\right)_{max}$};
         \path (C) ++ (1,-0.6) node {$\vec{F}_{max}$};
         \path (O) ++(1.5,1) node {$m$};
    \end{tikzpicture}
    \caption{Uma força aplicada no bloco e o atrito reagindo a força aplicada. Esse atrito é o atrito limite}
    \label{fig:atrito_estat_max}
\end{figure}

Para calcular essa força de atrito limite, usamos a seguinte equação:
\begin{equation}
    \left(F_{at}\right)_{max} = \mu_e\,N
\end{equation}
\noindent em que $\mu_e$ é o coeficiente de atrito estático (sem unidade) e N é o módulo da força normal $\vec{N}$.

\section{Atrito Cinético}

Quando um objeto desliza sobre um outro objeto/chão e temos um atrito envolvido, então o atrito será cinético: 
\begin{figure}[H]
    \centering
    \begin{tikzpicture}
         \coordinate (O) at (0,0);
         \coordinate (A) at (-3,0);
         \coordinate (B) at (5,0);
         \coordinate (C) at (3,2);
         
         \draw[black,ultra thick] (A) -- (B);
         \draw[black,thick] (O) rectangle (C);
         \draw[black,ultra thick,->] (3,1) -- ++(2,0);
         \draw[black,ultra thick,->] (O)+(0,.5) -- ++(-1.5,.5);
         \draw[black, ultra thick, ->] (O)+(1.5,0) --
         ++(1.5,-1.5);
         \draw[black, ultra thick, ->] (O)+(1.5,2) -- ++(1.5,3.5);
         \draw[black, ultra thick, ->] (3.6,2.9) -- ++(2,0);
         
         \path (O) ++ (4.5,3.2) node {$v=3\,m/s$};
         \path (O) ++ (1.5,3.9) node {$\vec{N}$};
         \path (O) ++ (1.5,-1.9) node {$\vec{P}$};
         \path (O) ++ (-0.9,1.1) node {$\vec{F}_{at}$};
         \path (C) ++ (1,-0.6) node {$\vec{F}$};
         \path (O) ++(1.5,1) node {$m$};
    \end{tikzpicture}
    \caption{Uma força aplicada no bloco, o bloco se movimentando e  o atrito reagindo a força aplicada.}
    \label{fig:atrito_cinet}
\end{figure}

A expressão do atrito cinético é dada por:
\begin{equation}
    F_{at} = \mu_c\,N
\end{equation}
\noindent em que $\mu_c$ é o coeficiente de atrito cinético e N é o módulo da força normal.

\section{Bloco num plano inclinado com atrito}

Vamos analisar um bloco num plano inclinado em que haja atrito entre o bloco e o plano:

\begin{figure}[H]
    \centering
    \begin{tikzpicture}
         \coordinate (O) at (0,0);
         \coordinate (A) at (-3,3);
         \coordinate (A1) at (-3,0);
         \coordinate (B) at (2,0);
         \coordinate (C) at (3,2);
         
         \draw[black,ultra thick] (A) -- (B);
         \draw[black, thick, dash pattern=on 3pt off 3pt] (A1) -- (B);
         \draw[black, thick, dash pattern=on 3pt off 3pt,->] (1,1) -- ++(60:2.5);
         \draw[black, thick, dash pattern=on 3pt off 3pt,->] (1,1) -- ++(-30:2.5);
         \draw[black, thick, dash pattern=on 3pt off 3pt] (A1) -- (B);
         \draw[black,thick, rotate around = {-32:(1,0,4)}] (-2.7,1.1) rectangle (-1.2,2.1);
         \draw[black,ultra thick,->] (-.75,2.2)--++(-.6,.4);
         \draw[black,ultra thick,->] (0.2,2.3) -- ++(.4,.6);
         \draw[black, ultra thick, ->] (-0.4,1.5) --
         ++(0,-1);
         \draw(B) +(180:1.5cm) arc (180:150:1.5cm);
         
         
         \path (O) ++ (.65,3.2) node {$\vec{N}$};
         \path (O) ++ (-.8,.7) node {$\vec{P}$};
         \path (O) ++ (-1.2,3.1) node {$\vec{F}_{at}$};
         \path (0.8,0.35) node {$\theta$};
         \path (1,1)+(55:2.5) node {$y$};
         \path (1,1)+(-24:2.5) node {$x$};
         
         \path (O) ++(-.2,1.8) node {$m$};
    \end{tikzpicture}
    \caption{Diagrama de forças de um bloco num plano inclinado com atrito. Os eixos coordenados foram escolhidos de forma a facilitar o problema}
    \label{fig:atrito_bloco}
\end{figure}

Perceba que, na direção  dos eixos coordenados, só a força peso não aponta para uma dessas direções. Por isso, precisaremos decompor a força peso nas 2 direções.

\begin{figure}[H]
    \centering
    \begin{tikzpicture}
         \coordinate (O) at (0,0);
         \coordinate (A) at (-3,3);
         \coordinate (A1) at (-3,0);
         \coordinate (B) at (2,0);
         \coordinate (C) at (3,2);
         
         
         \draw[black, thick, dash pattern=on 3pt off 3pt,->] (1,1) -- ++(60:2.5);
         \draw[black, thick, dash pattern=on 3pt off 3pt,->] (1,1) -- ++(-30:2.5);
         \draw(1,1) +(-120:0.6cm) arc (-120:-90:.6cm);
         
         
         \draw[black, ultra thick, ->] (1,1) --
         ++(0,-3);
         \draw[black, ultra thick, ->] (1,1) --
         ++(-120:2.5);
         \draw[black, ultra thick, <-] (1,-1.85) --
         ++(150:1.4);
         \draw (1,1)++(-120:2.1) -- ++(-30:.4) --++(-120:.4);
         \filldraw[fill=black] (1,1)++(-115:2.3) circle(1pt);
         
         \path (1.3,-.2) node {$\vec{P}$};
         \path (-.1,.2) node {$\vec{P}_y$};
         \path (.2,-1.9) node {$\vec{P}_x$};
         \path (1,1)+(55:2.5) node {$y$};
         \path (1,1)+(-24:2.5) node {$x$};
         \path (1,1)+(-105:1) node {$\theta$};
         
    \end{tikzpicture}
    \caption{Diagrama de forças de um bloco num plano inclinado com atrito. Os eixos coordenados foram escolhidos de forma a facilitar o problema. Esse ângulo $\theta$ é o mesmo ângulo de inclinação do plano inclinado.}
    \label{fig:atrito_bloco}
\end{figure}

Perceba que as 3 componentes ($P,\, P_x, \,P_y$) formam um triângulo retângulo, então podemos usar as relações trigonométricas conhecidas em relação ao ângulo $\theta$:
\begin{equation}
    \begin{split}
        sen\,\theta &= \frac{P_x}{P} \implies P_x = P\,sen\,\theta\\
        \cos\theta &=\frac{P_y}{P} \implies P_y = P\,\cos\theta
    \end{split}
\end{equation}

Com isso, o diagramas de força fica da seguinte forma:

\begin{figure}[H]
    \centering
    \begin{tikzpicture}
         \coordinate (O) at (0,0);
         \coordinate (A) at (-3,3);
         \coordinate (A1) at (-3,0);
         \coordinate (B) at (2,0);
         \coordinate (C) at (3,2);
         
         \draw[black,ultra thick] (A) -- (B);
         \draw[black, thick, dash pattern=on 3pt off 3pt] (A1) -- (B);
         \draw[black, thick, dash pattern=on 3pt off 3pt,->] (2,2) -- ++(60:1.5);
         \draw[black, thick, dash pattern=on 3pt off 3pt,->] (2,2) -- ++(-30:1.5);
         \draw[black, thick, dash pattern=on 3pt off 3pt] (A1) -- (B);
         \draw[black,thick, rotate around = {-32:(1,0,4)}] (-2.7,1.1) rectangle (-1.2,2.1);
         \draw[black,ultra thick,->] (-.75,2.2)--++(-.6,.4);
         \draw[black,ultra thick,->] (0.2,2.3) -- ++(60:0.8);
         \draw[black, ultra thick, ->] (-0.4,1.5) --
         ++(-120:0.8);
         \draw[black, ultra thick, ->] (0.6,1.5) --
         ++(-30:0.6);
         \draw(B) +(180:1.5cm) arc (180:150:1.5cm);
         
         
         \path (O) ++ (.65,3.4) node {$\vec{N}$};
         \path (O) ++ (-.8,.4) node {$\vec{P}_y$};
         \path (O) ++ (1.4,1.5) node {$\vec{P}_x$};
         \path (O) ++ (-1.2,3.1) node {$\vec{F}_{at}$};
         \path (0.8,0.35) node {$\theta$};
         \path (2,2)+(50:1.5) node {$y$};
         \path (2,2)+(-20:1.5) node {$x$};
         
         \path (O) ++(-.2,1.8) node {$m$};
    \end{tikzpicture}
    \caption{Diagrama de forças de um bloco num plano inclinado com atrito. Os eixos coordenados foram escolhidos de forma a facilitar o problema}
    \label{fig:atrito_bloco_diagrama}
\end{figure}
Com isso, percebemos que ao longo do eixo 'y', o bloco não se move, logo a força resultante tem que ser 0:
\begin{equation}
    (F_{res})_y = N - P_y = 0 \implies \boxed{N = P_y = P\cos\,\theta}
\end{equation}

Na direção x, a força resultante será:
\begin{equation}
    \begin{split}
        (F_{res})_x &= P_x - F_{at} = P\,sen\,\theta - \mu\,N\\
        &= P\,sen\,\theta - \mu\,P\,\cos\theta = mg(sen\,\theta - \mu\cos\theta)
    \end{split}
\end{equation}
Usando a $2^a$ Lei de Newton, a aceleração (a) é:
\begin{equation}
    m\,a_x = mg(sen\,\theta - \mu\cos\theta) \implies \boxed{a_x = g(sen\,\theta - \mu\cos\theta)}
\end{equation}

\section{Atrito entre blocos}

Um dos casos bem comuns num exercício sobre atrito é quando temos 2 blocos, um em cima do outro, em que existe atrito no contato entre os blocos. Vamos olhar o exemplo de 2 blocos empilhados deslizando em conjunto. A figura abaixo exemplifica:

\begin{figure}[H]
    \centering
    \begin{tikzpicture}
         \coordinate (O) at (0,0);
         \coordinate (A) at (-3,0);
         \coordinate (B) at (5,0);
         \coordinate (C) at (3,2);
         
         \draw[black,ultra thick] (A) -- (B);
         \draw[black,thick] (O) rectangle (C);
         \draw[black,thick] (0.5,2) rectangle (2.5,3);
         \draw[black,ultra thick,->] (3,1) -- ++(2,0);
         \draw[black,ultra thick,->] (O)+(0,.5) -- ++(-1.5,.5);
         \draw[black, ultra thick, ->] (O)+(2,0) --
         ++(2,-1.5);
         \draw[black, ultra thick, ->] (O)+(1,0) --
         ++(1,-1);
         \draw[black, ultra thick, ->] (O)+(1.5,3) --
         ++(1.5,4.5);
         \draw[black, ultra thick, ->] (3.5,4) --
         ++(1.5,0);
         
         
         
         \path (1.5,2.5) node {$m_2$};
         \path (O) ++ (2,-1.9) node {$\vec{P}_1$};
         \path (O) ++ (1,-1.4) node {$\vec{P}_2$};
         \path (O) ++ (1.5,4.8) node {$\vec{N}$};
         \path (O) ++ (-0.9,1.1) node {$\vec{F}_{at}$};
         \path (C) ++ (1,-0.6) node {$\vec{F}$};
         \path (O) ++(1.5,1) node {$m_1$};
         \path (O) ++(4.25,4.3) node {$\vec{v}$};
    \end{tikzpicture}
    \caption{Uma força aplicada no bloco debaixo. O conjunto se movimenta e o atrito reagindo a força aplicada.}
    \label{fig:atrito_2_corpos}
\end{figure}

Nós iremos determinar a força de atrito limite em que a massa $m_2$ não desliza em relação à massa $m_1$, sabendo que a massa $m_1$ é puxada por uma força $F$. Vamos olhar o diagrama de forças na massa $m_2$:

\begin{figure}[H]
    \centering
    \begin{tikzpicture}
         \coordinate (O) at (0,0);
         \coordinate (A) at (-3,0);
         \coordinate (B) at (5,0);
         \coordinate (C) at (3,2);
         
        
         \draw[black,thick] (0.5,2) rectangle (2.5,3);
         \draw[black,ultra thick,->] (2.5,2) -- ++(2,0);
         \draw[black,ultra thick,->] (1.75,3) -- ++(0,2);
         \draw[black,ultra thick,->] (1.75,2) -- ++(0,-2);
         
         \path (1.5,2.5) node {$m_2$};
         \path (O) ++ (1.75,-0.3) node {$\vec{P}_2$};
         \path (O) ++ (1.75,5.3) node {$\vec{N}_2$};
         \path (O) ++ (5.2,2) node {$\left(\vec{F}_{at}\right)_2$};
    \end{tikzpicture}
    \caption{Diagrama de forças na massa 2}
    \label{fig:atrito_m_2}
\end{figure}
Como o atrito entre os corpos é limite:
\begin{equation}
    \left(F_{at}\right)_2 = \mu_e\,N_2 \implies \boxed{\left(F_{at}\right)_2 = \mu_e\,m_2\,g}
\end{equation}
Usando a $2^a$ Lei de Newton, temos:
\begin{equation}\label{eq:a_x}
    m_2\,a_x = \left(F_{at}\right)_2 = \mu_e\,m_2\,g \implies \boxed{a_x = \mu_e\,g}
\end{equation}

Agora, analisaremos a força de atrito no bloco 1:

\begin{figure}[H]
    \centering
    \begin{tikzpicture}
         \coordinate (O) at (0,0);
         \coordinate (A) at (-3,0);
         \coordinate (B) at (5,0);
         \coordinate (C) at (3,2);
         
        
         \draw[black,thick] (0.5,2) rectangle (2.5,3);
         \draw[black,ultra thick,->] (2.5,2.5) -- ++(2,0);
         \draw[black,ultra thick,->] (1.75,3) -- ++(0,2);
         \draw[black,ultra thick,->] (1.75,2) -- ++(0,-2);
         \draw[black,ultra thick,->] (1.3,2) -- ++(0,-1.5);
         \draw[black,ultra thick,->] (.5,2) -- ++(-1.5,0);
         \draw[black,ultra thick,->] (0.5,3) -- ++(-1.5,0);
         
         \path (1.5,2.5) node {$m_1$};
         \path (O) ++ (1.75,-0.3) node {$\vec{P}_1$};
         \path (O) ++ (1.3,0.2) node {$\vec{N}_2$};
         \path (O) ++ (1.75,5.3) node {$\vec{N}_1$};
         \path (O) ++ (5.2,2.5) node {$\vec{F}$};
         \path (O) ++ (-1.3,2) node {$\vec{F}_{at}$};
         \path (O) ++ (-1.8,3.3) node {$\left(\vec{F}_{at}\right)_2$};
    \end{tikzpicture}
    \caption{Diagrama de forças na massa 1}
    \label{fig:atrito_m_1}
\end{figure}
Com isso, na vertical, temos:
\begin{equation}
    N_1 = N_2 + P_1 \implies N_1 = m_2\,g + m_1\,g \implies \boxed{N_1= (m_2+m_1)g}
\end{equation}
Na horizontal:
\begin{equation}
    (F_{res})_x = F - F_{at} - \left(F_{at}\right)_2
\end{equation}
Como o conjunto está se movimentando, o atrito dado por $F_{at}$ é cinético. Usando também o resultado obtido para $\left(F_{at}\right)_2$, temos que:
\begin{equation}
    \begin{split}
        (F_{res})_x &= F - \mu_{chao}\,N_1 - \mu\,m_2\,g\\
        &=F - \mu_{chao}\,(m_2+m_1)g - \mu\,m_2\,g
    \end{split}
\end{equation}
Pela $2^a$ Lei de Newton, temos que:
\begin{equation}
    m_1\,a_x = F - \mu_{chao}\,(m_2+m_1)g - \mu\,m_2\,g
\end{equation}
Mas, pela equação (\ref{eq:a_x}), a aceleração é $a_x = \mu\,g$, então:
\begin{equation}
    \begin{split}
        m_1\,\mu_e\,g &= F - \mu_{chao}\,(m_2+m_1)g - \mu\,m_2\,g\\
        F &= (m_1+m_2)\mu_e\,g + \mu_{chao}\,(m_2+m_1)g\\
    \end{split}
\end{equation}
Enfim:
\begin{equation}
    \boxed{F = g(m_1+m_2)(\mu_e+\mu_{chao})}
\end{equation}
\end{document}
