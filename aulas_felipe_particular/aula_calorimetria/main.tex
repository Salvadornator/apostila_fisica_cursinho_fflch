\documentclass[12pt]{extarticle}
%Some packages I commonly use.
\usepackage[portuguese]{babel}
\usepackage{graphicx}
\usepackage{framed}
\usepackage[normalem]{ulem}
\usepackage{amsmath}
\usepackage{amsthm}
\usepackage{amssymb}
\usepackage{amsfonts}
\usepackage{enumerate}
\usepackage[utf8]{inputenc}
\usepackage{float}
\usepackage{gensymb}
\usepackage[top=1 in,bottom=1in, left=1 in, right=1 in]{geometry}
\usepackage{multirow}
\usepackage{caption}
\usepackage{subcaption}
\usepackage[utf8]{inputenc}

%A bunch of definitions that make my life easier
\newcommand{\matlab}{{\sc Matlab} }
\newcommand{\cvec}[1]{{\mathbf #1}}
\newcommand{\rvec}[1]{\vec{\mathbf #1}}
\newcommand{\ihat}{\hat{\textbf{\i}}}
\newcommand{\jhat}{\hat{\textbf{\j}}}
\newcommand{\khat}{\hat{\textbf{k}}}
\newcommand{\minor}{{\rm minor}}
\newcommand{\trace}{{\rm trace}}
\newcommand{\spn}{{\rm Span}}
\newcommand{\rem}{{\rm rem}}
\newcommand{\ran}{{\rm range}}
\newcommand{\range}{{\rm range}}
\newcommand{\mdiv}{{\rm div}}
\newcommand{\proj}{{\rm proj}}
\newcommand{\R}{\mathbb{R}}
\newcommand{\N}{\mathbb{N}}
\newcommand{\Q}{\mathbb{Q}}
\newcommand{\Z}{\mathbb{Z}}
\newcommand{\<}{\langle}
\renewcommand{\>}{\rangle}
\renewcommand{\emptyset}{\varnothing}
\newcommand{\attn}[1]{\textbf{#1}}
\theoremstyle{definition}
\newtheorem{theorem}{Theorem}
\newtheorem{corollary}{Corollary}
\newtheorem*{definition}{Definition}
\newtheorem*{example}{Example}
\newtheorem*{note}{Note}
\newtheorem{exercise}{Exercise}
\newcommand{\bproof}{\bigskip {\bf Proof. }}
\newcommand{\eproof}{\hfill\qedsymbol}
\newcommand{\Disp}{\displaystyle}
\newcommand{\qe}{\hfill\(\bigtriangledown\)}
\newcommand{\grad}{$^\circ$}
\setlength{\columnseprule}{1 pt}
\usepackage[utf8]{inputenc}

\title{Aula 3 - Calorimetria}
\author{Felipe Salvador}
\date{Atualizado em \today}

\begin{document}

\maketitle

\section{Introdução}

Calorimetria é o estudo de como a transferência de energia térmica, mais conhecida como calor, acontece e quais são os processos dela.

Em geral, existem 2 tipos de calor:

\begin{itemize}
    \item \textbf{Calor sensível} - é o calor recebido ou dado por um corpo, de forma que aumente/diminua a temperatura, mas não altera a organização das moléculas dentro de um corpo;
    \item \textbf{Calor latente} - é o calor que um corpo recebe/dá em que traz uma mudança de como as moléculas do corpo se organizam nele. \textbf{Esse calor não altera a temperatura.} Uma questão importante desse calor é que ele é o responsável pelas mudanças de fases de um corpo (seja passar de sólido para líquido ou vice-versa, por exemplo). Esse calor é responsável pela solidificação, liquefação, gaseificação e assim por diante.
\end{itemize}

\section{Quantidades relacionadas ao calor}
\subsection{Capacidade Térmica (C)}

\textbf{A capacidade térmica é uma quantidade que mede a quantidade de calor necessária para que a temperatura do corpo cresça por 1.}

De forma geral, a capacidade térmica (C) é definido como:
\begin{equation}
    C = \frac{Q}{\Delta T} \implies Q = C\Delta T
\end{equation}
\noindent em que $Q$ é o calor que um corpo recebeu/deu e $\Delta T$ é a variação de temperatura que esse corpo sofreu. Como calor é um nome bonito para transferência de energia térmica, então a unidade de calor é a mesma unidade para energia.

No Sistema Internacional (SI), calor é dado em Joule(J). Mas, também, calor pode ser dado em Calorias (Cal), em que veremos mais tarde o porquê. Sabemos que temperatura é dado em graus Celsius (\grad C). Logo, pela fórmula:

\begin{equation}
    [C] = \frac{[Q]}{[\Delta T]} = \frac{J}{^\circ C} \quad \text{ou} \quad \frac{cal}{^\circ C}
\end{equation}

Essa quantidade é muito boa, pois é possível saber o quanto de calor precisa ser fornecido para um corpo aquecer. \textbf{Essa quantidade varia de corpo para corpo, de material para material.} 

\subsection{Calor específico (c)}

Quando temos um corpo formado por um único material, podemos saber o quanto calor 1g desse material precisa para aumentar/diminuir 1 \grad C. O nome dessa quantidade é \textbf{calor específico (c)}, e ele é dado como:

\begin{equation}
    c = \frac{Q}{m\Delta T} \implies Q = mc\Delta T
\end{equation}
\noindent em que $Q$ é o calor recebido/dado, $m$ é a massa do objeto e $\Delta T$ é o aumento/decrescimento da temperatura. Sabendo que a unidade de massa é dado em gramas (g), a unidade do calor específico é:

\begin{equation}
    [c] = \frac{[Q]}{[m][T]} = \frac{J}{g\,^\circ C} \quad ou \quad \frac{cal}{g\,^\circ C}
\end{equation}

A utilidade de usar $\frac{cal}{g\,^\circ C}$ é porque o calor específico da água é:
\begin{equation}
    c_{H_2O} = 1 \frac{cal}{g\,^\circ C} \approx 4,2 \frac{J}{g\,^\circ C}
\end{equation}

Importante ver, pelas equações (1) e (3) que, os calor específico e capacidade térmica são relacionadas por:

\begin{equation}
    C = mc
\end{equation}
\noindent mas detalhe: \textbf{essa relação só vale se o corpo for composto de um único material. Se o corpo tiver diversos materiais nele, essa relação não vale!}

\subsection{Calor Latente de Transformação (L)}

Essa quantidade diz o quanto de calor é necessário para que 1g de um certo material mude de estado (ex: sólido $\rightarrow$ líquido). Durante esse processo, lembremos que a temperatura não se altera, enquanto a transformação não acabar.

O calor latente de transformação é definido como:
\begin{equation}
    L = \frac{Q}{m} \implies Q =mL
\end{equation}
\noindent em que $Q$ é o calor recebido/dado pelo corpo e $m$ é a massa do corpo. Fazendo a mesma análise de unidade que fizemos para os outros, vemos que:
\begin{equation}
    [L] = \frac{[Q]}{[m]} = \frac{J}{g} \quad  ou \quad \frac{cal}{g}
\end{equation}

Por exemplo, o calor latente de vaporização da água (transformar água em vapor):
\begin{equation}
    L_{H_2O} = 540 cal/g
\end{equation}

\textbf{Importante: se o corpo estiver indo sólido $\rightarrow$ líquido, líquido $\rightarrow$ gás ou sólido $\rightarrow$ gás, L é positivo.}

\textbf{Se o corpo estiver indo líquido $\rightarrow$ sólido, gás $\rightarrow$ líquido ou gás $\rightarrow$ sólido, L é negativo.}

\section{Balanço de calor}

Uma questão importante é que: \textbf{o calor fornecido por um corpo é o calor recebido pelo outro corpo.} Isso é equacionado por:

\begin{equation}
    Q_{dado}^{total} + Q_{recebido}^{total} =0
\end{equation}
\noindent em que o $Q_{dado}^{total}$ é o todo calor dado por um corpo,  e $Q_{recebido}^{total}$ é todo calor que o outro corpo recebeu. A última, é a soma dos calores recebidos (que são dados pelas equações (1), (3) e (7)).

É a partir dessa relação, que iremos fazer os nossos exemplos:

\section{Exemplos:}

\begin{enumerate}
    \item Foi dado 20000 cal de calor para 1000 g de água inicialmente a 20\grad C. Encontre a temperatura final para essa massa de água. A água evaporou?
    (Dados: calor específico da água: 1 cal/g\grad C; calor latente de vaporização: 540 cal/g; calor específico de vapor: 0,5 cal/g\grad C)
    
    Usando a fórmula dada por (10) para o balanço de calor:
    \begin{align*}
        &Q_{dado}^{total} + Q_{recebido}^{total} =0
        \end{align*}
        Como é água, o calor recebido deve aumentar a temperatura:
        \begin{align*}
        &Q_{dado}^{total} + mc\Delta T =0\\
        &-20000 + 1000. 1.(T_F - 20) =0 \\
        &1000T_F - 20000 - 20000 =0\\
        &1000T_F = 40000 \implies T_F = 40\,^\circ C
        \end{align*}
        
        Bem, chegamos na resposta. Mas, pode ser que precisemos colocar outros termos (calor latente, por exemplo) para dar conta.
        
        \item Os mesmos dados do exemplo 1, mas agora a massa de água é de 100g.
        
        O balanço de calor é dado por
        \begin{align*}
        &Q_{dado}^{total}+ Q_{recebido}^{total} =0
        \end{align*}
        Como é água, o calor recebido deve aumentar a temperatura:
        \begin{align*}
        &Q_{dado}^{total} + mc\Delta T =0\\
        &-20000 + 100. 1.(T_F - 20) =0 \\
        &100T_F - 20000 - 2000 =0 \\
        &100T_F = 22000 \implies T_F = 220 ^\circ C
        \end{align*}
        
        A gente sabe que a água evapora a 100 \grad C, então essa resposta acima está errada. Logo, no balanço de calor, temos que contabilizar a vaporização da água.
        
        \begin{align*}
            &Q_{dado}^{total} + mc\Delta T - mL =0\\
            &-20000 + 100.1(100-20) + 100.540 =0
            &-20000 + 8000 +54000 =0
        \end{align*}
        
        Isso claramente não dá 0. Mas o que acontece aqui é que parte da massa de água vaporizou, uma outra parte ainda está líquida!
        
        \item Tendo uma massa de água de 100 g, inicialmente a 20 \grad C. A água foi colocada no congelador e sua temperatura final é de -5 \grad C. Qual é o calor cedido ao congelador? (Dado: calor específico do gelo: 0,5 cal/g \grad C; calor latente de fusão: 80 cal/g)
        
        Pelo balanço de energia:
        
        \begin{align*}
            &Q_{dado}^{total} + Q_{recebido}^{total} =0
        \end{align*}
        
        Nós vamos olhar da perspectiva do congelador. Então:
        \begin{align*}
             Q_{dado} + mc_{água}\Delta T + mL + mc_{gelo}\Delta T =0
        \end{align*}
        Nós sabemos que a água vira gelo à 0 \grad C, então:
        \begin{align*}
            &Q_{dado} + 100.1.(0-20) - 100.80 + 100.0,5 (-5-0) =0\\
            &Q_{dado} -2000 - 8000 -250 =0\\
            &Q_{dado} = 10250 cal
        \end{align*}
        
        Lembrar que $L$ é negativo, porque a água tá se tornando gelo.
\end{enumerate}

\end{document}
