\documentclass[12pt]{extarticle}
%Some packages I commonly use.
\usepackage[english]{babel}
\usepackage{graphicx}
\usepackage{framed}
\usepackage[normalem]{ulem}
\usepackage{amsmath}
\usepackage{amsthm}
\usepackage{amssymb}
\usepackage{amsfonts}
\usepackage{enumerate}
\usepackage[utf8]{inputenc}
\usepackage{float}
\usepackage{gensymb}
\usepackage[top=1 in,bottom=1in, left=1 in, right=1 in]{geometry}
\usepackage{multirow}
\usepackage{caption}
\usepackage{subcaption}
\usepackage[utf8]{inputenc}

%A bunch of definitions that make my life easier
\newcommand{\matlab}{{\sc Matlab} }
\newcommand{\cvec}[1]{{\mathbf #1}}
\newcommand{\rvec}[1]{\vec{\mathbf #1}}
\newcommand{\ihat}{\hat{\textbf{\i}}}
\newcommand{\jhat}{\hat{\textbf{\j}}}
\newcommand{\khat}{\hat{\textbf{k}}}
\newcommand{\minor}{{\rm minor}}
\newcommand{\trace}{{\rm trace}}
\newcommand{\spn}{{\rm Span}}
\newcommand{\rem}{{\rm rem}}
\newcommand{\ran}{{\rm range}}
\newcommand{\range}{{\rm range}}
\newcommand{\mdiv}{{\rm div}}
\newcommand{\proj}{{\rm proj}}
\newcommand{\R}{\mathbb{R}}
\newcommand{\N}{\mathbb{N}}
\newcommand{\Q}{\mathbb{Q}}
\newcommand{\Z}{\mathbb{Z}}
\newcommand{\<}{\langle}
\renewcommand{\>}{\rangle}
\renewcommand{\emptyset}{\varnothing}
\newcommand{\attn}[1]{\textbf{#1}}
\theoremstyle{definition}
\newtheorem{theorem}{Theorem}
\newtheorem{corollary}{Corollary}
\newtheorem*{definition}{Definition}
\newtheorem*{example}{Example}
\newtheorem*{note}{Note}
\newtheorem{exercise}{Exercise}
\newcommand{\bproof}{\bigskip {\bf Proof. }}
\newcommand{\eproof}{\hfill\qedsymbol}
\newcommand{\Disp}{\displaystyle}
\newcommand{\qe}{\hfill\(\bigtriangledown\)}
\setlength{\columnseprule}{1 pt}
\usepackage[utf8]{inputenc}

\title{Calorimetria e Dilatação}
\author{Felipe Salvador}
\date{Setembro 2019}

\begin{document}

\maketitle

\section*{Introdução}

Começamos o último assunto do ano, que é Termodinâmica. Ela estuda as relações de troca de energia, temperatura, pressão e volume de um sistema, por meio de processos, como calor e trabalho.

Nessa primeira parte, vamos só nos reter ao estudo dos processos que só envolvem calor, chamado de \textbf{Calorimetria}. Olharemos para processos entre objetos, inicialmente, com temperaturas diferentes e veremos como esse processo evolui até o fim, que damos o nome de \textbf{Equilíbrio Térmico}.

Na segunda parte, veremos o que acontece com um corpo quando aquecemos ele (damos calor). Ele esquenta, mas também ele pode mudar a sua forma e tamanho. Esse fenômeno damos o nome de \textbf{Dilatação}.

\section{Calorimetria}

Calorimetria é a área de estudo sobre como a energia térmica de corpos (quão quentes/frios esses corpos estão) flui, quando esses corpos são postos em contato direta ou indiretamente.

O conceito mais importante desse estudo é o conceito de \textbf{Calor}.

\subsection{O que é Calor?}

\textbf{Calor é a troca de energia térmica de um corpo para outro corpo.} 

\end{document}
