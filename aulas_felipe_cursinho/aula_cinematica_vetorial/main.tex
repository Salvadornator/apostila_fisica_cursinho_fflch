\documentclass[12pt]{extarticle}
%Some packages I commonly use.
\usepackage[portuguese]{babel}
\usepackage{graphicx}
\usepackage{framed}
\usepackage[normalem]{ulem}
\usepackage{amsmath}
\usepackage{amsthm}
\usepackage{amssymb}
\usepackage{amsfonts}
\usepackage{enumerate}
\usepackage[utf8]{inputenc}
\usepackage{float}
\usepackage{gensymb}
\usepackage[top=1 in,bottom=1in, left=1 in, right=1 in]{geometry}
\usepackage{multirow}
\usepackage{caption}
\usepackage{subcaption}
\usepackage[utf8]{inputenc}
\usepackage{tikz}

%A bunch of definitions that make my life easier
\newcommand{\matlab}{{\sc Matlab} }
\newcommand{\cvec}[1]{{\mathbf #1}}
\newcommand{\rvec}[1]{\vec{\mathbf #1}}
\newcommand{\ihat}{\hat{\textbf{\i}}}
\newcommand{\jhat}{\hat{\textbf{\j}}}
\newcommand{\khat}{\hat{\textbf{k}}}
\newcommand{\minor}{{\rm minor}}
\newcommand{\trace}{{\rm trace}}
\newcommand{\spn}{{\rm Span}}
\newcommand{\rem}{{\rm rem}}
\newcommand{\ran}{{\rm range}}
\newcommand{\range}{{\rm range}}
\newcommand{\mdiv}{{\rm div}}
\newcommand{\proj}{{\rm proj}}
\newcommand{\R}{\mathbb{R}}
\newcommand{\N}{\mathbb{N}}
\newcommand{\Q}{\mathbb{Q}}
\newcommand{\Z}{\mathbb{Z}}
\newcommand{\<}{\langle}
\renewcommand{\>}{\rangle}
\renewcommand{\emptyset}{\varnothing}
\newcommand{\attn}[1]{\textbf{#1}}
\theoremstyle{definition}
\newtheorem{theorem}{Theorem}
\newtheorem{corollary}{Corollary}
\newtheorem*{definition}{Definition}
\newtheorem*{example}{Example}
\newtheorem*{note}{Note}
\newtheorem{exercise}{Exercise}
\newcommand{\bproof}{\bigskip {\bf Proof. }}
\newcommand{\eproof}{\hfill\qedsymbol}
\newcommand{\Disp}{\displaystyle}
\newcommand{\qe}{\hfill\(\bigtriangledown\)}
\setlength{\columnseprule}{1 pt}
\usepackage[utf8]{inputenc}

\usetikzlibrary{arrows,shapes,positioning,angles}
\usetikzlibrary{decorations.markings}
\tikzstyle arrowstyle=[scale=1]
\tikzstyle directed=[postaction={decorate,decoration={markings,
    mark=at position 1 with {\arrow[arrowstyle]{stealth}}}}]
\tikzstyle reverse directed=[postaction={decorate,decoration={markings,
    mark=at position 1 with {\arrowreversed[arrowstyle]{stealth};}}}]

\title{Aula 19 - Cinemática Vetorial (Movimento Oblíquo e Lançamento)}
\author{Felipe Salvador}
\date{Atualizado em \today}

\begin{document}

\maketitle

\section{Introdução}

Nessa aula, veremos como analisamos como um objeto se movimenta em 2 dimensões. O método que usaremos para a resolução desse tipo de movimento é tentar escrever o sistema em 2 direções: x e y.

Faremos a decomposição do movimento nessas 2 direções e relacionaremos o movimento movimento nessas direções com o movimento geral a partir da trigonometria. Por fim, deduziremos relações importantes do lançamento oblíquo.

\section{Caso Geral - Lançamento de uma bola a partir do chão}

Para estudar o caso geral, vamos usar o exemplo do lançamento de uma bola. Inicialmente, a bola está no chão e na origem de nosso sistema de coordenadas (ou seja, $x=0$ e $y=0$). Damos um chutão na bola, de forma que a bola começa a fazer o movimento com uma velocidade $\vec{v}_0$. 

Vamos sempre considerar que a bola está na Terra, ou seja, ela está sujeita a força da gravidade e, por consequência, à aceleração da gravidade. Também, iremos desconsiderar a resistência do ar e processos dissipativos. Essa velocidade inicial faz um ângulo $\theta$ com o chão. O movimento que a bola faz está na imagem abaixo:

\begin{figure}[H]
    \centering
    \begin{tikzpicture}

    % define coordinates
    \coordinate (O) at (0,0) ;
    \coordinate (A) at (0,4) ;
    \coordinate (B) at (-4,0);
    \coordinate (C) at (4,0)
    
    % media

    % axis

    % rays
    \filldraw[fill=green!20!,draw=green!20!] (-6,0) rectangle (7,-4.1);
    \draw[black,ultra thick,directed] (B) -- +(1.09,2);
    \draw[black,directed,ultra thick] (A) -- +(1.09,0);
    \draw[black,directed,ultra thick] (C) -- +(1.09,-2);
    
    \draw[black, ultra thick, directed] (6,4) -- +(0,-2);
    
    \draw[black!75!,dash pattern=on5pt off 3pt,thick, directed] (B) --+(1.09,0);
    \draw[black!75!,dash pattern=on5pt off 3pt, thick,directed] (B) --+(0,2);
    
    \draw[black!75!,dash pattern=on5pt off 3pt,thick, directed] (C) --+(1.09,0);
    \draw[black!75!,dash pattern=on5pt off 3pt, thick,directed] (C) --+(0,-2);
    \draw[black!75!,thick, <->] (A)++(0,-0.2) -- (0,0);
    \draw [black,dash pattern=on 5pt off 3pt,thick,->] (-4,4)--(-4,5);
    \draw [black,dash pattern=on 5pt off 3pt,thick,->] (-4,4)--(-3,4);
    
    \draw[blue,thick, dash pattern=on 5pt off 3pt] (B) parabola [bend pos=0.5] bend +(0,4) +(8,0);
    
    \draw (B) +(0:0.7cm) arc (0:57:0.7cm);
    \draw (C) +(0:0.7cm) arc (0:-57:0.7cm);
    \path (B) ++(30:0.9) node{$\theta$};
    \path (C) ++(-30:0.9) node{$\theta$};

    % angles
    \filldraw[fill=blue] (B) circle (5pt);
    \filldraw[fill=blue] (A) circle (5pt);
    \filldraw[fill=blue] (C) circle (5pt);
    \node[] at (6.3,3)  {$\vec{g}$};
    \path (B) ++ (-0.5,0.2) node {\textbf{A}};
    \path (A) ++(-.3,-0.5) node {\textbf{B}};
    \path (C) ++(-0.5,0.2) node {\textbf{C}};
    \path (B) ++(50:1.9) node {$\vec{v}_0$};
    \path (B) ++(-0.5,1.8) node {$v_{0y}$};
    \path (B) ++(1,-0.3) node {$v_{0x}$};
    \path (C) ++(-50:1.9) node {$\vec{v}_0$};
    \path (C) ++(-0.5,-1.8) node {$v_{0y}$};
    \path (C) ++(1,0.3) node {$v_{0x}$};
    \path (A) ++(0.7,0.4) node {$\vec{v}_{0x}$};
    \path (A) ++(0.3,-2) node {$h$};
    \path (-4.3,5) node {$y$};
    \path (-3,3.6) node {$x$};
\end{tikzpicture}
    \caption{Esquema do lançamento de uma bola. O ponto A é o ponto inicial do movimento, o ponto B é o ápice do movimento (quando a bola está no ponto mais alto do movimento) e o ponto C é o ponto em que a bola volta a tocar no chão.}
    \label{fig:refracao_1}
\end{figure}

\subsection{Decomposição vetorial}
Para fazermos a análise do movimento, teremos que fazer a decomposição vetorial da velocidade. Pela figura anterior, vimos que a velocidade inicial $\vec{v}_0$ faz um ângulo $\theta$ com a horizontal. Com isso, nós podemos, por meio de uma soma vetorial, decompor o vetor $\vec{v}_0$ em termos de um vetor na direção $x$, $\vec{v}_{0x}$, mais um vetor na direção $y$, $\vec{v}_{0y}$.

Com isso, se definir o eixo $x$ na horizontal e o eixo $y$ na vertical:
\begin{figure}[H]
    \centering
    \begin{tikzpicture}

    % define coordinates
    \coordinate (O) at (0,0) ;
    \coordinate (A) at (0,4) ;
    \coordinate (B) at (-4,0);
    \coordinate (C) at (4,0)
    
    % media

    % axis

    % rays
    \draw[black,ultra thick,->] (O) -- +(3,3);
    
    
    \draw[black!75!,dash pattern=on5pt off 3pt,thick, ->] (O) --+(3,0);
    \draw[black!75!,dash pattern=on5pt off 3pt, thick,->] (O)++(3,0) --+(0,2.85);
    \draw[black,dash pattern=on5pt off 3pt, thick,->] (B) --+(2,0);
    \draw[black,dash pattern=on5pt off 3pt, thick,->] (B) --+(0,2);
    
    \draw (O) +(0:0.7cm) arc (0:45:0.7cm);
    \path (O) ++(25:0.95) node{$\theta$};
    \draw (O)++(2.5,0)-- +(0,0.5) --+(0.5,.5);
    \draw (B)++(.5,0)-- +(0,0.5) --+(-.5,.5);
    \filldraw[fill=black] (O)++(2.75,.25) circle(1pt);
    \filldraw[fill=black] (B)++(.25,.25) circle(1pt);

    % angles
    \filldraw[fill=blue] (0) circle (5pt);
    \path (O) ++(55:1.9) node {$\vec{v}_0$};
    \path (O) ++(3.5,1.5) node {$\vec{v_{0y}}$};
    \path (O) ++(1.5,-0.3) node {$\vec{v_{0x}}$};
    \path (B) ++(2,-0.5) node{$x$};
    \path (B) ++(-0.3,2) node{$y$}
\end{tikzpicture}
    \caption{Decomposição do vetor velocidade $\vec{v}_0$ nas direções $x$ e $y$.}
    \label{fig:decomposicao}
\end{figure}

Perceba que esse conjunto de vetores formam um triângulo retângulo. O $\vec{v_{0y}}$ é o cateto oposto, $\vec{v_{0x}}$ é o cateto adjacente e $\vec{v}_0$. Logo, podemos usar as relações trigonométricas para determinar qual é o módulo de $\vec{v}_{0y}$ e $\vec{v}_{0x}$:
\begin{align*}
    sen\,\theta &= \frac{v_{0y}}{v_0} \implies v_{0y} = v_0\,sen\,\theta\\
    \cos\theta &= \frac{v_{0x}}{v_0} \implies v_{0x} = v_0\,\cos\theta
\end{align*}

\subsection{Fórmulas e análise dos movimentos}
Com as expressões da velocidade inicial nas direções x e y, podemos analisar o movimento nas 2 direções. \textbf{Como não estamos considerando a resistência do ar, não há nenhuma força aplicada na direção x, logo, a aceleração na direção x é zero.} Com isso, \textbf{o movimento na direção 'x' é um movimento retílineo uniforme (MRU).}

Já na direção 'y', só há uma força sendo aplicada nessa direção: a força peso. Pela Segunda Lei de Newton, portanto, há uma aceleração na direção 'y', que é a \textbf{aceleração da gravidade ($\vec{g}$). Por causa dessa aceleração, o movimento na direção 'y' é um movimento uniformemente variado (MUV).}

Essa discussão vale para qualquer caso de lançamento oblíquo!! Com isso, as equações horárias do movimento são:
\begin{equation}\label{eq:1}
    \begin{split}
        y &= y_0 + v_{0y}\,t + \frac{(-g)t^2}{2}\quad\quad (MUV)\\
        x &= x_0 + v_{0x}\,t\quad\quad (MRU)
    \end{split}
\end{equation}
\noindent em que $x,\,y$ é a posição do objeto no instante 't', $x_0,\,y_0$ é a posição inicial dele, $v_{0x},\,v_{0y}$ é a velocidade inicial do objeto nas direções 'x' e 'y' e $g$ é aceleração da gravidade.

Com as informações iniciais ($x_0,\,y_0,\,v_{0x},\,v_{0y}$), dado um instante 't', sabemos a posição exata do objeto na trajetória (x,y). Além da fórmula do Sorvetão, podemos usar as outras fórmula do MUV para o movimento na direção 'y', como:
\begin{equation}
    \begin{split}
        v_y^2 &= (v_{0y})^2 + 2\,(-g)\,\Delta y\quad\quad \text{(Equação de Torricelli)}\\
        v_y &= v_{0y} + (-g)\,t \quad\quad \text{(Equação da velocidade)}
    \end{split}
\end{equation}

Com esse conjunto de equação dado por (\ref{eq:1}), podemos finalmente usar as fórmulas do $v_{0x}$ e $v_{0y}$, obtidas acima para escrever todas as equações que podemos usar:
\begin{equation}
    \boxed{\begin{split}
        &\left.\begin{array}{lcr}
            y = y_0 + v_0\,sen\,\theta\,t - \frac{gt^2}{2}\\
        v_y^2 = v_0^2\,sen^2\theta -2g\Delta y\\
        v_y = v_{0}\,sen\, \theta -gt\\
        \end{array}\right\} \quad\text{Equações para o eixo y}
        \\
        &\left.\begin{array}{lcr}
            x = x_0 + v_{0}\cos\theta\, t 
        \end{array}\right\}\quad \text{Equação para o eixo x}
    \end{split}}
\end{equation}

A partir da equação de Torricelli, podemos deduzir a altura máxima de um lançamento a partir da velocidade inicial e o ângulo do lançamento. Lembrando que na altura máxima, a velocidade instantânea $v_y$ é 0. Logo:
\begin{equation}
    \begin{split}
        v_y^2 &= 0 = v_0^2\,sen^2\theta -2g\,\Delta y \\
        2g\,\Delta y &= v_0^2\,sen^2\theta \implies \boxed{h =\Delta y = \frac{v_0^2\, sen^2\theta}{2g}}
    \end{split}
\end{equation}

Se fizermos um lançamento puramente vertical ($\theta =90^\circ \implies sen\,90^\circ = 1$), então a fórmula se reduz a:
\begin{equation}
    h = \Delta y = \frac{v_0^2}{2g}
\end{equation}

Podemos também calcular o alcance de um lançamento, ou seja, qual é a distância horizontal que o objeto no movimento oblíquo fez até chegar no chão de novo:
\begin{figure}[H]
    \centering
    \begin{tikzpicture}

    % define coordinates
    \coordinate (O) at (0,0) ;
    \coordinate (A) at (0,4) ;
    \coordinate (B) at (-4,0);
    \coordinate (C) at (4,0);
    
    % media

    % axis

    % rays
    \filldraw[fill=green!20!,draw=green!20!] (-6,0) rectangle (7,-4.1);
    \draw[black,ultra thick,directed] (B) -- +(1.09,2);
    \draw[black,directed,ultra thick] (A) -- +(1.09,0);
    \draw[black,directed,ultra thick] (C) -- +(1.09,-2);
    
    \draw[black, ultra thick, directed] (6,4) -- +(0,-2);
    
    
    
    \draw[black!75!,thick, <->] (A)++(0,-0.2) -- (0,0);
    \draw[black!75!,thick, <->] (B)++(0.2,0) -- (3.8,0);
    \draw [black,dash pattern=on 5pt off 3pt,thick,->] (-4,4)--(-4,5);
    \draw [black,dash pattern=on 5pt off 3pt,thick,->] (-4,4)--(-3,4);
    
    \draw[blue,thick, dash pattern=on 5pt off 3pt] (B) parabola [bend pos=0.5] bend +(0,4) +(8,0);
    
    % angles
    \filldraw[fill=blue] (B) circle (5pt);
    \filldraw[fill=blue] (A) circle (5pt);
    \filldraw[fill=blue] (C) circle (5pt);
    \node[] at (6.3,3)  {$\vec{g}$};
    \path (B) ++ (-0.5,0.2) node {\textbf{A}};
    \path (A) ++(-.3,-0.5) node {\textbf{B}};
    \path (C) ++(-0.5,0.2) node {\textbf{C}};
    \path (B) ++(50:1.9) node {$\vec{v}_0$};
    \path (C) ++(-50:1.9) node {$\vec{v}_0$};
    \path (A) ++(0.7,0.4) node {$\vec{v}_{0x}$};
    \path (A) ++(0.3,-2) node {$h$};
    \path (-4.3,5) node {$y$};
    \path (-3,3.6) node {$x$};
    \path (0,-0.3) node {$A$}
\end{tikzpicture}
    \caption{Diagrama do movimento oblíquo de uma bola com as principais quantidades. 'h' é altura do ápice do movimento e 'A' é o alcance do lançamento.}
    \label{fig:Alcance}
\end{figure}
Para calcular o alcance, precisamos achar o instante 't' em que o objeto volta a tocar no chão, ou seja, quando $y=0$. Para isso, vamos usar a fórmula do Sorvetão, tomando a altura inicial $y_0=0$:
\begin{align*}
    y &= y_0 + v_0\,sen\,(\theta) t -\frac{gt^2}{2}\\
    0&= v_0\,sen\,(\theta) t -\frac{gt^2}{2}\\
    &t(v_0\,sen\,\theta - \frac{gt}{2}) =0 \implies t=0 \quad ou \quad v_0\,sen\theta - \frac{gt}{2} =0\\
    &\implies \frac{gt}{2} = v_0 \,sen\,\theta \implies \boxed{t = \frac{2v_0\,sen\,\theta}{g}}
\end{align*}
Agora que sabemos o instante em que o objeto volta ao chão, vamos encontrar a posição $x$ no instante que o objeto toca o chão, ou seja, vamos calcular a posição $x$ no instante $t = \frac{2v_0\,sen\,\theta}{g}$ (tomando a posição x inicial como $x_0=0$):
\begin{align*}
    x &= x_0 + v_0\cos\theta t\\
    x &= v_0\,\cos\theta \left(\frac{2\,v_0\,sen\,\theta}{g}\right)\\
    x &= 2\,sen\,\theta\,\cos\theta\,\frac{v_0^2}{g}
\end{align*}
Usando a fórmula do seno $sen\,2\theta = 2\,sen\,\theta\,\cos\theta$, finalmente temos a fórmula do Alcance:
\begin{equation}
    \boxed{A = x = \frac{v_0^2\,sen\,2\theta}{g}}
\end{equation}
Para uma velocidade inicial definida, o maior alcance é quando o seno é máximo, que é quando: $sen\,2\theta = 1$. Mas, o seno é 1 quando o ângulo é de $90^\circ$, logo:
\begin{equation}
    \boxed{2\theta_{max} = 90^\circ \implies \theta_{max} = 45^\circ}
\end{equation}

\textbf{Então, para termos o alcance máximo de um lançamento, o ângulo do lançamento deve ser de $45^\circ$.}

\end{document}

